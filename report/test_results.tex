\section{Test results}
\label{section:test_results}
In this section I will present the test results. I will start by discussing my test methodology, then the hardware/software specifications, and then I will show the test results themselves, and then round off by an analysis of the results.

\subsection{Spanner test results}
This section describes some of the important results I have found in the limited-range spanner test. The distance/hops results can be found in section \ref{limited_range_spanner_distance_appendix} and the neighbour results can be found in section~\ref{limited_range_spanner_neighbour_appendix}.

First of all we must check whether or not the results are worth checking in the first place. From table~\ref{table:analysis_50} through table~\ref{table:analysis_10000} we can see that we are dealing with at most $1.32$ different connected components and that in no case have we been able to detect if the either the \ac{gabe} or the \ac{rng} has cut off all paths between two nodes, if it existed in the \ac{ucg}. 

The relatively low average number of connected components is a good indicator that the returned tests are of decent quality, as there must have been many different paths, of varying length, to choose from.

As expected, since they all follow the \ac{uga}\footnote{See section~\ref{section:apply_uga} on p. \pageref{section:apply_uga}.}, if there is a path between two nodes in the \ac{ucg}, then a path can also be found in the \ac{gabe} or the \ac{rng}, since all the ``error'' counts are zero in the tables.

Since one of the stark secondary differences between the planar and the \acp{ucg} is the differences in neighbours, I have chosen to illustrate the development of this compared to the average minimum number of hops between two nodes\footnote{It is important to note that the shortest is not necessary the path that is going to be used, depending on the chosen routing algorithm.} in section~\ref{section:hop_neighbour_comparison} and figure~\ref{graph:avg_neighbour}. Furthermore, I have compared the average distance and the average number of hops in section~\ref{section:graph_distance_comparison} and figure~\ref{graph:dist_percent}, since these are the values that will let us test whether there may be any spanner-like relationship between the \ac{ucg}, the \ac{gabe} and the \ac{rng}.

\subsubsection{Hop/neighbour comparison}
\label{section:hop_neighbour_comparison}
\graphResult{spanner/avg_neighbour}{The solid lines indicates the average number neighbours (left axis). The dashed lines indicates the average number of hops from the source to the sink (right axis).}{avg_neighbour}

As we can see from figure~\ref{graph:avg_neighbour}, it is clear that the average number of the smallest number of hops from the source to the sink increases as the number of nodes in the graph increases. It is also clear that it is the \ac{ucg} which has the smallest average number of hops required, while the \ac{rng} has the largest. Comparing these results with the average number of neighbours it is however very interesting to see that while the average number of neighbours keeps increasing for the \ac{ucg}, it is completely stable for the \ac{rng} and, except for the number of neighbours for 50 and 100 nodes, the \ac{gabe}, which agrees with the discussion in section~\ref{del_gabe_rng_neigh}. 

\subsubsection{Graph Distance Comparison}
\label{section:graph_distance_comparison}

\graphResult{spanner/dist_percent}{The solid lines indicates the percentage of the total distance traversed for the \ac{gabe} and the \ac{rng}, compared to the \ac{ucg}, while the dashed lines indicates the same, but for hops instead of Euclidean distance. The \ac{ucg} has been omitted as it in both cases would be a straight line at $100\%$.}{dist_percent}

From figure~\ref{graph:dist_percent} we can see that not surprisingly that the \ac{gabe} is closer to the \ac{ucg} is much better than the \ac{rng}. More importantly we also see that there is a huge difference between the Euclidean distance than the Unit distance, when compared to the results from the \ac{ucg}. It is interesting to note, that even though the density of the graphs are kept steady, as the number of nodes grows, both the \ac{gabe} and the \ac{rng} edges asymptotically closer to the \ac{ucg}. 

\subsection{}







\subsection{Test analysis}


I have performed these


\subsection{}
\label{section:test_results_spanners}
