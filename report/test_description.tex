\section{Test description}
In this section I will describe which metrics I am going to use to compare the different algorithms. Having specified these I will describe the tests in detail.

\subsection{Metrics}

The metrics I will measure in the tests will be 
\begin{description}
\item[Number of hops:] The number of hops from the source to the sink. For this I will measure the maximum number of hops, the minimum number of hops, as well as the average. I feel the number of hops is a useful metric, for multiple reasons. Firstly it tells us whether the algorithm might lead the message astray \todo{add more} . Secondly, the more hops a message performs, the more energy has to be expended on routing it, cutting down down on the life-time of the network.

\item[Time:] The amount of time spent sending the message from the source to the sink. I will likewise measure the maximum, minimum and average time spent. It is clear that the faster a message arrives at its sink, the better.

\item[Number of misses:] The ratio of timeouts compared to the number of sent messages. This will apply whether we are using UDP or TCP. \todo{make this more precise}

\item[Percentages of successfully arrived messages:] While most\todo{qualify/quantify this} routing algorithms guarantees that the message will always arrive, this is not always the case. The best example of a routing algorithm that does not have this guarantee is the greedy routing algorithm (see Section~\ref{greedy} p. \pageref{greedy}). Also, since we are trying to simulate mobile nodes with limited energy storage, critical parts of the topology may be fail, making it impossible for the message to arrive.
\end{description}

For the number of hops, the amount of time and number of misses I will record the maximum, the minimum and the average value. The main value for comparison will be the average value, but I feel that recording both the maximum and the minimum values will give me a view of both the extremes, but also gives me additional information about the average.

\subsection{Input parameters}
In order to perform the tests I will need to define the parameters that are going to vary for the different tests.
\begin{description}
\item[Movement model:] The topology is clearly going to be influenced by the way that the nodes move, therefore it would be interesting to test out several models. 

\item[Routing Algorithms:] 

\item[Amount and data-transmission type:] In real world situations there will be different levels of traffic on the network, and therefore to test it we must likewise simulate these differences. Directly effecting this is the protocol used to send the data through the network. Applications that uses the UDP protocol is less likely to cause congestion in the network, compare to TCP packages.

\item[Size of the simulation area:] The size of the simulation area has a direct influence on how far apart the nodes can move and how many nodes are needed for a given node density. Everything else being equal, a larger simulation area will make the network less robust.

\item[Number of nodes:] It is clear that denser node distribution, everything else being equal, will give a more robust network,

\item[Percentage of nodes failing/leaving the network:] Since we are dealing with a \manet, it is clear that a nodes may leave the network or fail (either due to equipment failure or lack of power). 
\item[Percentage of nodes entering the network:] In \manet nodes can also enter the network, something we need to simulate if we are to get an adequate picture of the algorithms worth.
\end{description}


\subsection{Actual parameters}
I now that I have detailed which kinds of parameters we can change 

\begin{itemize}
\item[Movement models:] DisasterArea \cite{disasterArea},  Random Street \cite{randomStreet}, and Gauss-Markow. In practise I will create a trace using the BonnMotion tool \cite{toilers} use to create the movement files which can then be imported into ns-2.
\item[Routing algorithms:] GOAFR \cite{gopher}, GOAFR+ \cite{gopher+}, Greedy \cite{gopher}, DSDV \todo{find article and refer to it}
\item[Number of nodes:] 500, 1000\footnote{Neither GloMoSim or ns-2 can handle that many nodes without extensive \todo{find and cite their problems}}
\item[Nodes failing:] 0\%, 10\%, 25\%, 40\%
\item[Nodes entering:] 0\%, 5\%, 25\%
\end{itemize}
