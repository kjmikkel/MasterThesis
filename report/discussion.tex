\section{Discussion}
\label{section:discussion}

\subsection{Choice of routing graph}

From the results presented in section~\ref{section:test_results}, it becomes clear that our assumption about the ad- and disadvantages of the three different types are correct. The \ac{ucg} is the one needing the fewest hops and the smallest distance on average to go from the source to the sink, but has far more neighbours than the other two. The \ac{gabe} lies in the middle between the \ac{ucg} and the \ac{rng} with regards to distance/hops, while the \ac{rng} has the fewest neighbours, but also the longest distance/hops. 

The question is then which of these graphs would be optimal in a \ac{manet}. As mentioned earlier the problem with having many neighbours is not the cost in transmitting (that value is fixed by using the \ac{uga}), but rather the bookkeeping involved in remembering all the neighbours, and picking one when sending a message. Given this I would argue that having an average 10 or more neighbours are not acceptable, when alternatives with only around 2 or 3 exist. However, it is also clear that there is a clear cost of minimum number of hops when looking at figure~\ref{graph:avg_neighbour} and figure~\ref{graph:dist_percent}, which clearly shows us that the \ac{gabe} and \ac{rng} requires far more hops on average to get from the source to the sink than the \ac{ucg}. Given their planar nature, this is not surprising. 

Based on this I would recommend using the \ac{gabe}, since it reduces the average number of neighbours with an average that, regardless of the number of nodes in the graph, lies very close to 3 with a very small standard deviation\footnote{See table~\ref{table:neigh_50} through table~\ref{table:neigh_10000}.}, while having requiring fewer hops on average compared to the \ac{rng}, as seen in both figure~\ref{graph:avg_neighbour} figure~\ref{graph:dist_percent}.

\subsection{Spanner-like properties of the \ac{gabe} and the \ac{rng}}

\subsubsection{Euclidean distance}
\graphResult{spanner/euclid_distance}{A closer look at the euclidean distance compared to the \ac{ucg}. Uses the same data as figure~\ref{graph:dist_percent}}{euclid_distance}

In figure~\ref{graph:dist_percent} and figure~\ref{graph:euclid_distance} we clearly see that the Euclidean distance for both graphs tops in the beginning, and then begins to flat out. One should think that this would be a good argument for it being a spanner, but since the standard deviation increases with the number of nodes, this is not entirely clear.

\subsubsection{Hop distance}

\graphResult{spanner/unit_distance}{A closer look at the Unit distance compared to the \ac{ucg}. Uses the same data as figure~\ref{graph:dist_percent}}{unit_distance}

From figure~\ref{graph:dist_percent} and figure~\ref{graph:unit_distance} we can clearly see from the number of hops (the dashed lines), that they seem to be slowly converging. However, we can also see that if we consult table~\ref{table:analysis_50}-\ref{table:analysis_10000} that the standard deviation is also increasing as the number of nodes increases (and clearly to the advantage of the average maximum values). However, since something very similar is happening to the \ac{ucg}, this is somewhat lessened.

\subsubsection{Spanner conclusion}

While I think that it would be too strong based on the above to conclude that the \ac{gabe} and \ac{rng} forms limited-range spanners to the limited \ac{ucg}, I do think it is safe to conclude from the above that for uniformly placed nodes in the plane, both the euclidean and the Unit distance on average are so close to the spanner (even for large numbers), that they in practice can be relied upon.

\subsection{}
