\section{Routing Algorithms}

The biggest conceptual difference for routing/shortest path algorithms in a \manet compared to main-stray routing/shortest path algorithms, is the loss of global overview that these algorithms have --- such as relaxing a specific nodes' edges at a specific time. There exist distributed version of some of the ``classic'' routing algorithms --- such as Bellman-Ford --- but their usefulness is limited in a mobile network, since they will have to be recomputed either each time a node wants to transmit a message, or each time the underlying graph changes.

As mentioned in Section~\ref{fundamental}, the problem of sending a message from the source to the sink  becomes a graph routing problem \footnote{But not necessary a shortest-path routing problem, since factors like fairness or maximisation of the lifetime of the \manet becomes issues in practise}.

When routing a message from the source to the sink, it is important to distinguish the message that is being transmitted and the nodes that are transmitting. The transmitted message does not only include the information that is intended for the node at the sink, but also extra meta-data needed for routing. This includes time-to-live, position of the source node, position of the sink, message id etc. The nodes are all assumed to use the same routing algorithm, and at the very least knowing who their neighbours are.

Most of the following algorithms will mostly be based around the idea of \emph{local routing algorithms}. 

Paraphrasing \cite{compass}, a local routing algorithm is an algorithm that for fills the following criteria:
\begin{enumerate}
\item Each message will have a limited memory to store information about a constant number of vertexes, but it have full access to the position of the sink and the source. At no point does the message have full knowledge of the entire graph. 
\item Let $s$ be the source of the message, $t$ the sink, and $v$ be a node in the graph that can be reached from $s$. Arriving at $v$, the message can then use the information $v$ has stored about its neighbours. Based on this information, one of the neighbours of $v$ is chosen as the next receiver of the message, unless $v = t$, in which case the message will not be propagated.
\item A message will not leave information that the nodes routing algorithm will be able to use. In this sense nodes are considered stateless.\end{enumerate}

Combined, the first and third property says that the state of routing process is contained in the message being routed, while the control and logic of the routing is contained in the nodes, which only acts upon the information in the message received and their knowledge about their neighbours. Since all the nodes are running the same routing algorithm, a well designed routing algorithm ensures that the message will arrive.

\label{record-recived}
It is worth noting that the 3 properties only apply to the routing algorithm itself, and not the layers below. This is an important distinction, as most routing solutions require reach node to track which messages it receives, so that future copies of the same message can be ignored in the future --- ensuring that a message will not needlessly congest the network. In the same vain, some routing solutions, such as \cite{speed} has its lower layers use this information to apply congestion avoidance. Another application is to use the messages being passed around to keep track of which nodes are still active --- this is usually combined with diagnostic or ``accounting'' messages\todo{Find ref}. 

\subsection{Categories of Routing algorithms}
When a routing algorithm wants to send a message or a number of messages from a sink to a source, an important question is whether it has been maintaining a path to the sink, it has to create a path to the sink, or each message should find its own way from the source to the sink.

If the algorithm requires and already established path, then there are two fundamentally different principles that can be applied.
\begin{description}
\route{Destination Sequential Distance Vector (DSDV)}
{With the DSDV, every node tries to remember as many as possible paths to the other nodes in the graph. The mapping from nodes to paths can either just be the next neighbour that a message needs to be routed to, or the entire path. In pure implementations of DSDV, each node knows a path to all other nodes. The advantage of DSDV is that when the information is correct it gives an efficient path, while the disadvantage is that it is memory and bandwidth intensive, since each node has to contain a full routing table, and thus has problems scaling to large \manet. Depending on how much the nodes move, it could also be prone to contain outdated information.
%In order to find a path DSDV uses a distributed version of Bell
}
{DSDV}

\route{Ad-Hoc On-demand Distance Vector (AODV)}
{With the AODV, a node only records a path from any pair of nodes if it becomes relevant. This is done by a path discovery phase, where a route request package (RREQ) is sent from the source. When the RREQ arrives at a node it checks whether the node knows of a path to the sink. If the nodes does not, then the RREQ records the node it was sent from and the number of hops to the current node, and proceeds. If the current node does know a path to the sink, then a bidirectional path is established from the source to the sink, so messages can be exchanged.

The AODV has route maintenance mechanism, that works thus: Let node $n_k$ and $n_{k+1}$ be the [$k$]th and [$k+1$]th node on the path from the source to the sink for $k \in \N$. If $n_{k+1}$ is no longer adjacent to $n_{k}$, then it will begin a tear-down protocol that will delete its routing table, and then tell the [$k-1$]th node to do the same until the message arrives at the source, which can then rebuild the path if it is still relevant.
AODV does not describe any specific routing algorithm to establish a route.}
{larMANET, AODV}

\end{description}

\tikfig{greedy-problem}{In this topology we attempt greedy routing from node \emph{s} to node \emph{t}. The edges between the nodes define a bidirectional link. The grey background indicates the chosen path. We see that the greedy algorithm first routes the message to node \emph{a}, since node \emph{a} is closer to node \emph{t} than node \emph{d}. However, since neither node \emph{b} or node \emph{c} are closer to node \emph{t} than node \emph{a} (as illustrated by the radius centred on node \emph{t}), the greedy routing does not continue. Node \emph{a} is in this situation often called the \emph{local minimum}. In this context, the face created by node \emph{a}, \emph{b}, \emph{c} and node \emph{t} a \emph{void}}

Routing algorithms suited for \manet s can be sorted into the following categories:
\begin{description}
\route{Flooding}
{The simplest of the routing algorithms. The entire network is flooded with the message, ensuring that the message eventually reaches the sink. We can be sure that the message will not flood the network forever, thanks to the recording system detailed in section \ref{record-recived}. If the sink can be reached from the source, then flooding is  guarantied to work, but will use a lot of the networks bandwidth, and requires many of the nodes to be active.}
{Sur2, larMANET} 

\tikfig{rdr}{The Grey area is the area in which routing take place. It is clear that the shape of the flooding will have a large effect on the number of nodes that will receive the transmission.}

\route{Restricted Direction Routing (RDR)}
      {Restricted Directed Routing is an algorithm that can be described as a kind of directed flooding. Let G be a geometrical shape. G is now placed such that both the source and the sink are both contained in G (preferably with a non-zero margin, to counter act any movement). The source now floods all nodes inside G with the message, hopefully reaching the sink. Once the first message arrives, a path between the source and sink is established, and any future messages in that pack is sent through this path \todo{clarify}.

The there is no standard shape, but possible choices includes Linear Swept Sphere, Circle or Rectangle -- see figure \ref{fig:rdr}. Obviously, RDR is only applicable when the source has an idea of the location of the sink, otherwise another algorithm (like flooding) has to be employed. In the case where both nodes wants to contact each other simultaneously (say, in situations where timers are involved), a bi-directional RDR can take place, where each node tries to find each other, until both reach each other, and a path can be established.}
{larMANET}

\route{Directed Diffusion (DD)}
{Directed Diffusion is a pull algorithm that works by having the sink send out a request for information. This requested information constitute a \emph{name} and can both be specific or be in a range\footnote{For instance it could be about data for a single day, or a range of days.}. This name, together with meta-data such as the sinks location and time-stamp, comprises an \emph{interest} which is transmitted to the rest of the network. Depending on the implementation, the request can flood the entire network, or just a section of it (like in RDR). When a node receives a request it checks with its data cache whether it has the information. If the node has the information, it sends a confirmation message back to the node it got the interest from. If the node does not have the information, it stores the interest in its own cache and creates a \emph{gradient} to both keep track of where the interest came from, have a timeout mechanism for that link, and a path maintenance mechanism (that also takes care of the interval between transmissions for that particular interest). 

Once a confirmation message arrives at the sink, the sink will perform \emph{reinforcement} that will use the established network of gradients to create one or more paths to the source, and begin to have the requested data transmitted over these paths. As a sink begins to receive data from a path, it will increase the flow of data from that particular path, leading to a positive feedback-loop that will eventually choose a single path over the others.

Since DD is a pull algorithm it might be especially useful in sensor networks.}
{directed}

\route{Greedy routing} 
{\label{greedy}Greedy routing is a simple routing scheme that always chooses to route the message to that path of its neighbours that is closest to the sink. In the case where the message arrives at a node $n$, where all of $n$'s neighbours are farther away from the sink than $n$, then no action is specified, and the message will not be delivered -- see figure~\ref{fig:greedy-problem} for an illustration. The greedy algorithm thus offers no guarantee in delivering the message, but in practise it improves as the density of the network increases.}
{gopher, beyondUnit}
\route{Geometrical routing}
{Geometrical routing uses the location of the current node and the location of the sink. There exist many different kinds of routing algorithms, but common for them is that unlike the greedy routing technique, they sometimes not transmit the message to a node that is closest to the sink. This flexibility allows the geometrical algorithms to find a path to the sink. An example of a Geometrical algorithm could be the OFR \cite{gopher} routing algorithm, which uses the line-segments between the sink and the source to find which faces in the graph it needs to route to eventually arrive at the sink}
{gopher+, gopher, gpsr}

\route{Shortest path} 
      {The shortest path technique uses distributed versions of shortest-path algorithms in order to find shortest-path from either one node to another, or from all nodes to all other nodes. As one might imagine, this approach has problems in mobile networks, where the structure of the graph can be in constant flux. Even in static networks an all-to-all solution will use $O(n^2)$ for the whole system where $n$ is the number of nodes. Even if this is only $O(n)$ for each node, this is still rather expensive. Examples of distributed algorithms include Distributed Bellman-Ford}
{Sur2, contractionGraph, DSDV}

\end{description}

Different routing algorithms make different assumptions about the mobility of the graph. There are several routing algorithms \cite{adaptive, two-tier} that the infrastructure contains a number of stationary nodes or sensors that can send and receive information. 

\tikfig{hidden_terminal}{The hidden terminal problem. In this example node c is communicating with node b (illustrated by the grey colour in node b's transmitting radius) Since the transmission does not reach node a, it is not possible for node a to detect the communication between node b and node c, and it will thus believe that it is free to transmit its message to node b. For more detailed information, see \cite{ComNet} or similar.}

In practise the system will have to deal with all the problems inherent in wireless communication, such as the hidden terminal problem (see figure~\ref{fig:hidden_terminal}) and message collision. It would be preferable if a lower communications layer took care of this instead of having to implementing it in the algorithm itself.

\subsection{Fundamental Graph Routing problems}

A fundamental problem that has to be answered in any geographical routing algorithm is now to deal with situations where a node has sent a message to a sink, but does not know the sinks location. There are several solutions to this problem:

\begin{itemize}
\item Once the sink has moved a certain distance away from its neighbours, it could inform them of its current position, and then tells its local neighbours. Effectively this solution creates an area of way-points that can be used to find the sink \todo{add reference}.
\item If we are dealing with a cluster method (see Section~\ref{cluster methods} for explanation of clusters), then the sink only needs to inform its old \ch that it has moved to a new \ch, and in the same way as before, the message will catch up with the sink.
\item Instead of getting the exact location of the sink at the beginning and then routing blindly towards that location, layers of node location ``servers'', really just a normal node that knows the location of several nodes in the network, can be established. A server may not know where any given node is, but it may know where a closer server that either knows the location of the sink, or knows a server closer to the sink \ldots and so on until we find a server that knows where the sink is. Thus the sink will eventually be found \todo{add reference}.
\end{itemize}

Another problem found in Graph routing is that of security. Unless some kind of security system is employed, a malicious node can not only impersonate other nodes, but it can also sniff the messages that are transmitted in its area (not only the ones that are directly sent to it). Further a malicious node can spam the network with bogus messages, creating congestion in the network and draining the nodes power. One proposed solution to this problem is to include secure keys that uniquely identify each node. However, without a central key-server (both for distribution and authentication), this is not practical \todo{ref to security article}.

\section{Problems we have with movement}

Movement presents us with the following problems:

\tikfig{shortest_path_change}{The nodes that are connected with an edge have an bidirectional link, and the bidirectional links with a grey background are the ones that are routed through. In \ref{fig:short_path1} we see a path from node $+$ to node $*$. In \ref{fig:short_path2} we see that the movement of node $-$ means that we have to chose for longer path to the node $*$ from node $+$.}

\begin{itemize}
\item Changes underlying graph -- see figure~\ref{fig:shortest_path_change}.
\item Can invalidate location information about the node \todo{make figures for all of these}
\item If the node is part of path from \emph{a} to \emph{b} (and the node is neither), which is stored and which is believed to be used again, then that path may fail, and a longer path may have to be taken. In degenerate cases this may lead to situations where the new optimal path will lead away from the absolute coordinates of the end node.
\item The moving node may partition a network into 2 separate networks, if the node was the only node the 2 networks could communicate through.
\item The moving node may create a link between 2 previously separate networks, creating the need for location exchanges for the 2 other disjoint networks.
\item The node may move into a position that creates a better path than the one before it moved --- so the surrounding nodes must be informed about this.
\end{itemize}

The movement creates the need for the following solutions:
\begin{itemize}
\item There needs to be a way to restructure the underlying graph so that it reflects the movement of the node \cite{practical}
\item The location of the moving node needs to be updated. This does not necessary need to be done to all the nodes in the network, nor all the nodes that knew it before. There are many schemes for this, such as GLS \cite{scaleLocation} which partitions the world into squares, and then partitions these squares into other squares, resulting in several layers of squares. In each of these squares, there is a single node that knows the location of the node (and a scheme to find it) \todo{add reference}.
\item Depending on implementation there may be no problem --- a time-out system that detects that the message did not arrive, the node right before the missing node reports back that it could not deliver the message etc. The system would then have to find a new path to the sink.
\item If the network is partitioned, then the best course of action is to have a mechanism that will ensure that will allow the nodes to forget about the nodes they are no longer connected to. 
\item If connection is established between network \emph{a} and network \emph{b}, there should be a mechanism so that network \emph{a} and inform network \emph{b} of its nodes, and vice versa. Optimally this should already be a part of how nodes advertise their presence.
\item In order to take advantage of new nodes, they need to advertise their presence (and their location) to the entire system 
\end{itemize}
